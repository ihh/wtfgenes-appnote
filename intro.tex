
Term Enrichment Analysis (TEA) is a common technique for finding functional patterns,
specifically over-represented ontology terms, in a set of experimentally identified genes \citep{pmid15297299}.
The most common approach,
which we refer to as {\em Frequentist TEA},
is a one-tailed Fisher's Exact Test (based on the hypergeometric distribution,
which models the number of term-associations if the gene set was chosen by chance),
with a suitable correction for multiple hypothesis testing.
Frequentist TEA has been implemented many times on various platforms \citep{pmid12431279,pmid11829497,pmid12702209,pmid15297299,pmid18511468,pmid22543366,pmid23868073,pmid23586463}.

A model-based alternative to Frequentist TEA, which more directly addresses multiple testing issues
(for example, by modeling the partitioning of an observed gene set by complementary terms),
is {\em Bayesian TEA}.
In contrast to Frequentist TEA, which just rejects a null hypothesis that genes are chosen by chance,
Bayesian TEA explicitly models the alternative hypothesis that the gene set was generated by a few random ontology terms.
This approach was introduced by \cite{pmid18676451}, developed by \cite{pmid20172960},
and implemented in Java and R \citep{pmid21561920}.
However, the model-based approach remains significantly less well-explored than Frequentist TEA.

\begin{figure}
\includegraphics[width=\columnwidth]{model}
\caption{
  \label{fig:model}
  Model-based explanation of observed genes ($O_i$) using ontology terms ($T_j$), following \cite{pmid20172960}.
  Other variables and hyperparameters are defined in the text.
  Circular nodes indicate continuous-valued variables or hyperparameters;
  square nodes indicate discrete-valued (boolean) variables.
  Dashed lines indicate deterministic relationships;
  shaded nodes indicate observations.
  Plates (rounded rectangles) indicate replicated subgraph structures.
}
\end{figure}


The graphical model underpinning Bayesian TEA is sketched in Figure~\ref{fig:model}.
For each of the $m$ terms there
is a boolean random variable
$T_j$ (``term $j$ is activated'').
For each of the $n$ genes there is a directly-observed boolean random variable
$O_i$ (``gene $i$ is observed in the gene set''),
and one deterministic boolean variable
$H_i$ (``gene $i$ is activated'')
defined by $H_i = 1 - \prod_{j \in G_i} (1 - T_j)$
where $G_i$ is the set of terms associated with gene $i$
(including directly annotated terms, as well as ancestral terms implied by transitive closure of the directly annotated terms).
The probability parameters are $\pi$ (term activation), $\alpha$ (false positive) and $\beta$ (false negative),
and the respective hyperparameters are ${\bf p}=(p_0,p_1)$, ${\bf a}=(a_0,a_1)$ and ${\bf b}=(b_0,b_1)$.
The model is
\begin{eqnarray*}
P(T_j=1|\pi) & = & \pi \\
P(O_i=1|H_i=0,\alpha) & = & \alpha \\
P(O_i=1|H_i=1,\beta) & = & 1-\beta
\end{eqnarray*}
with
$\pi \sim \mbox{Beta}({\bf p})$,
$\alpha \sim \mbox{Beta}({\bf a})$ and
$\beta \sim \mbox{Beta}({\bf b})$.
The model of \cite{pmid20172960} is similar but used an
{\em ad hoc} discretized prior for $\pi$, $\alpha$ and $\beta$.


Most Bayesian and Frequentist TEA implementations are designed for desktop use.
Several Frequentist TEA implementations are designed for the web, such as
DAVID-WS \citep{pmid22543366} and
Enrichr \citep{pmid23586463} % \citep{pmid23586463,pmid25971742,pmid27141961}
which has a rich dynamic web front-end.
However, these generally require a server-hosted back end that executes code.
Further, there are no web-based Bayesian TEA implementations.
