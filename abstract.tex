
% Abstract
\structabs{
A common technique for interpreting experimentally-identified lists of genes
is to look for enrichment of genes associated with particular ontology terms.
The most common test uses the hypergeometric distribution;
more recently, a model-based test was proposed.
These approaches must typically be run using downloaded software, or on a server.
}{
We develop a collapsed likelihood for model-based gene set analysis and present \wtfgenes, an implementation of both hypergeometric and model-based
approaches, that can be published as a static
site with computation run in JavaScript on the user's web browser client.
Apart from hosting files, zero server resources are required: the site can (for example) be served
directly from Amazon S3 or GitHub Pages.
A C++11 implementation yielding identical results runs roughly twice as fast as the JavaScript version.
}{
\wtfgenes\ is available from \url{https://github.com/evoldoers/wtfgenes} under the BSD3 license.
A demonstration for the Gene Ontology is usable at \url{https://evoldoers.github.io/wtfgo}.
}{
Ian Holmes {\tt ihholmes+wtfgenes@gmail.com}.
}{
None.
}
