% Definitions
\newcommand\wtfgenes{WTFGenes}

% Abstract
\structabs{
A common technique for interpreting experimentally-identified lists of genes
is to look for enrichment of genes associated to particular Gene Ontology terms.
The most common technique uses the hypergeometric distribution;
more recently, a model-based approach was proposed.
These approaches must typically be run using downloaded software, or on a server.
}{
We present \wtfgenes, a reference implementation of hypergeometric and model-based
Gene Ontology enrichment analysis, that can be published server-free as a static
site with computation run in JavaScript on the user's web browser client.
No server resources are required, though a faster C++ implementation yielding identical
results is also provided.
Our implementation of model-based Gene Ontology enrichment uses some optimizations
which permit probability parameters to be integrated out directly.
}{
\wtfgenes\ is available from \url{https://github.com/evoldoers/wtfgenes}.
}{
Ian Holmes {\tt ihholmes+wtfgenes@gmail.com}.
}{
None.
}

\section*{Introduction}

Gene Set Enrichment Analysis (GSEA) \cite{pmid16199517}
Numerous implementations e.g. GO::TermFinder \cite{pmid15297299}

Model-based Gene Set Analysis (MGSA) \cite{pmid20172960}
Bioconductor \cite{pmid21561920}

builds on earlier generative model by \cite{pmid18676451}

% description of MGSA from Bauer et al 2010

Common feature of MGSA and GSEA implementations is that they are designed either for desktop use
or for server-hosted applications

\section*{Results}

Optimization of MGSA: direct integration over parameters

Static site implementation

Speed comparison

\section*{Discussion}

GREAT \cite{pmid20436461}

\section*{Funding}

IHH was partially supported by NHGRI grant R01-HG004483.

\bibliographystyle{natbib}
%\bibliographystyle{bioinformatics}
%\bibliographystyle{achemnat}
%\bibliographystyle{plainnat}
%\bibliographystyle{abbrv}
%
%\bibliographystyle{plain}
%
%\bibliography{Document}


\bibliography{references}
